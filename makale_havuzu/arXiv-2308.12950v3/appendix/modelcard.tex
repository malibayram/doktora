\newpage\clearpage

\section{Model card}

\Cref{tab:modelcard} 
presents a model card \citep{mitchell2019modelcard} for the family of models we release.

% For any changes please bear in mind that they also have to end up in the repo.

\begin{table}[h!]
{
\small
    \centering
    \begin{tabular}{p{0.18\textwidth}|l}
    \thickhline
    \multicolumn{2}{c}{\bf Model details}\\
    \thickhline
        Model Developers & Meta AI 
        \\\hline
        Variations & \makecell[lp{0.75\textwidth}]{
        \model comes in four model sizes, and three variants:  the base \model, \pymodel designed specifically for Python and \instmodel~for instruction following and safer deployment. 
        All variants are available in sizes of 7B, 13B, 34B and 70B parameters.
        }\\\hline
        Input&Models input text only.
        \\\hline
        Output&Models output text only.
        \\\hline
        Model Architecture&\makecell[lp{0.75\textwidth}]{
        \model and its variants are autoregressive language models using optimized transformer architectures. \model 7B, 13B and 70B additionally support infilling text generation. All models but \pymodel~70B and \instmodel~70B were fine-tuned with up to 16K tokens, and support up to 100K tokens at inference time.
        }\\\hline 
        % Model Dates&\makecell[lp{0.75\textwidth}]{
        % \model and its variants have been trained between January 2023 and July 2023.
        Model Dates&\makecell[lp{0.75\textwidth}]{
        \model and its variants have been trained between January 2023 and January 2024.
        } 
        \\\hline 
        Status&\makecell[lp{0.75\textwidth}]{
        This is a static model trained on an offline dataset.
        Future versions of \instmodel will be released  as we improve model safety with community feedback.}
        \\\hline 
        Licence&\makecell[lp{0.75\textwidth}]{
        A custom commercial license is available at: \url{ai.meta.com/resources/models-and-libraries/llama-downloads/}. 
        }\\\hline 
        \makecell[lp{0.18\textwidth}]{Where to send comments}&\makecell[lp{0.75\textwidth}]{
        Instructions on how to provide feedback or comments on the model can be found in the model README, or by opening an issue in the GitHub repository (\url{https://github.com/facebookresearch/codellama/}).}
        \\
        \thickhline
        \multicolumn{2}{c}{\bf Intended Use}\\
        \thickhline
        Intended Use Cases&\makecell[lp{0.75\textwidth}]{
        \model and its variants are intended for commercial and research use in English and relevant programming languages. 
        The base model \model can be adapted for a variety of code synthesis and understanding tasks, \pymodel is designed specifically to handle the Python programming language, and \instmodel 
        is intended to be safer to use for code assistant and generation applications. 
        }\\\hline 
        Out-of-Scope Uses&\makecell[lp{0.75\textwidth}]{
        Use in any manner that violates applicable laws or regulations (including trade compliance laws). Use in languages other than English. Use in any other way that is prohibited by the Acceptable Use Policy and Licensing Agreement for \model and its variants.
        }\\
        \thickhline
        \multicolumn{2}{c}{\bf Hardware and Software}\\
        \thickhline
        Training Factors&\makecell[lp{0.75\textwidth}]{
        We used custom training libraries. The training and fine-tuning of the released models have been performed Meta’s Research Super Cluster.
        }\\\hline 
        Carbon Footprint&\makecell[lp{0.75\textwidth}]{
        In aggregate, training all 12 \model models required 1400K GPU hours of computation on hardware of type A100-80GB (TDP of 350-400W). Estimated total emissions were 228.55 tCO2eq, 100\% of which were offset by Meta’s sustainability program.
        }\\
        \thickhline
        \multicolumn{2}{c}{\bf Training Data}\\
        \thickhline
        \multicolumn{2}{c}{
        \makecell[lp{0.95\textwidth}]{
        All experiments reported here and the released models have been trained and fine-tuned using the same data as \llamavtwo \citep{touvron2023llamav2} with different weights (see Section~\ref{sec:method} and Table~\ref{tab:dataset}).
        \instmodel uses additional instruction fine-tuning data.
        }
        }\\\thickhline 
        \multicolumn{2}{c}{\bf Evaluation Results}\\
        \thickhline
        \multicolumn{2}{c}{
        \makecell[lp{0.95\textwidth}]{See evaluations for the main models and detailed ablations \Cref{sec:results} and safety evaluations \Cref{sec:safety}.
        }}
        \\\thickhline
        \multicolumn{2}{c}{\bf Ethical Considerations and Limitations}
        \\\thickhline
        \multicolumn{2}{c}{
        \makecell[lp{0.95\textwidth}]{\model and its variants are a new technology that carries risks with use. Testing conducted to date has been in English, and has not covered, nor could it cover all scenarios. For these reasons, as with all LLMs, \model’s potential outputs cannot be predicted in advance, and the model may in some instances produce inaccurate or objectionable responses to user prompts. Therefore, before deploying any applications of \model, developers should perform safety testing and tuning tailored to their specific applications of the model. Please see the Responsible Use Guide available available at \url{https://ai.meta.com/llama/responsible-user-guide}.}}
        \\
        \thickhline
    \end{tabular}
    \caption{Model card for \model.}
    \label{tab:modelcard}
}
\end{table}

