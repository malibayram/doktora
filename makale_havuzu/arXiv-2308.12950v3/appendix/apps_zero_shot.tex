\section{Zero shot results on APPS}
\label{appendix:apps_zero_shot}

In addition to two-shot results we report in Table \ref{tab:apps_res}, we also list the zero-shot performance for \instmodel in Table \ref{tab:apps_zero_shot}. For both the two-shot and zero-shot results, we use nucleus sampling ($p$ = 0.95) at temperature 0.6 for all of our models. The prompt templates are shown in \ref{fig:prompt_for_apps}. We prompt the model to wrap the final code answer inside of triple single quotes, which makes it easier to extract the answer. We use a special instruction to help models understand the specific question format: ``read from and write to standard IO'' for standard questions and ``use the provided function signature'' for call-based questions, which we insert into our prompt as the question guidance. Despite not finetuned on the training data nor provided with few shot examples, \instmodel can achieve convincing results on these challenging competitive programming questions. 

\begin{table}[t!]
  \center
   \setlength{\tabcolsep}{3pt}
  \begin{tabular}{rccccccccc} %ccr@{}}
  \toprule
  Size & \multicolumn{3}{c}{Introductory} & \multicolumn{3}{c}{Interview} & \multicolumn{3}{c}{Competition} \\
  & Pass@5 & Pass@10 & Pass@100 & Pass@5 & Pass@10 & Pass@100 & Pass@5 & Pass@10 & Pass@100 \\
    \midrule
7B  & \acc{24.85} & \acc{29.40} & \acc{41.26} & \acc{6.26} & \acc{8.40} & \acc{16.07} & \acc{1.94} & \acc{3.03} & \acc{9.15} \\
13B & \acc{24.77} & \acc{29.80} & \acc{43.54} & \acc{6.99} & \acc{9.24} & \acc{17.31} & \acc{1.69} & \acc{2.52} & \acc{6.33} \\
34B & \acc{19.81} & \acc{25.94} & \acc{43.53} & \acc{5.68} & \acc{7.98} & \acc{16.90} & \acc{1.51} & \acc{2.29} & \acc{6.36} \\
  \bottomrule
  \end{tabular}
  \caption{\textbf{\instmodel APPS zero shot results.} All results are calculated with raw outputs without any filtering. \label{tab:apps_zero_shot}}
\end{table}
