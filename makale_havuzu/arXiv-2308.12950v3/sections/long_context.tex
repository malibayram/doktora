Effective handling of long sequences is a major topic of research in transformer-based language modeling~\citep{vaswani2017attention}.
The fundamental modeling challenges are extrapolation, i.e., operating on sequence lengths beyond those seen at training time, and the quadratic complexity of attention passes which favors training on short-to-medium length inputs.

For \model, we propose a dedicated \emph{long context fine-tuning (LCFT)} stage in which models are presented with sequences of 16,384 tokens, up from the 4,096 tokens used for \llamavtwo and our initial code training stages.
By limiting the training time spent on processing long sequences to a fine-tuning stage, we gain long-range capabilities without significantly increasing the cost of training our models.
Our strategy is similar to the recently proposed fine-tuning by position interpolation~\citep{chen2023extending},  and we confirm the importance of modifying the rotation frequencies of the rotary position embedding used in the \llamavtwo foundation models \citep{su2021roformer}.
However, instead of downscaling frequencies linearly as \citet{chen2023extending}, we change the base period from which they are derived.
Specifically, with rotary embeddings, the query and key vectors $\mathbf{x}_n$ at position $n$ are subject to a linear transformation $\mathbf{R}^d_{\Theta,n} \mathbf{x}_n$, where $\mathbf{R}^d_{\Theta,n}$ is a block diagonal matrix with entries of the form
\begin{align*}
   \left( \mathbf{R}^d_{\Theta,n} \right) _i  =
   \begin{pmatrix}
   \cos n \theta_i & -\sin n \theta_i \\
   \sin n \theta_i & \cos n \theta_i
   \end{pmatrix},
\end{align*}
and $d$ denotes the embedding dimension.
Rotation frequencies are computed as $\theta_i = \theta^{-2i/d}$, and we increase the base period $\theta$ from 10,000 to 1,000,000 for fine-tuning.
This increase allows for processing much larger sequences and reduces bias towards short-distance attention (see \Cref{app:lcft_details} for further discussion).
Our experiments confirm that \model models are not only effective within the increased sequence length used during fine-tuning, but further show extrapolation capabilities and exhibit stable behavior on very long sequences of up to 100,000 tokens (\Cref{sec:results-lcft}).