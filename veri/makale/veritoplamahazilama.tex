Veri toplama süreci, doktorların hastalara verdiği cevapları içeren doktor profilleri ve soru-cevap verilerini içeren bir veri seti üzerinde gerçekleştirilecektir. Bu veri seti, doktorsitesi.com adlı bir web sitesinden elde edilmiştir. Bu web sitesi, doktorların doğrulanmış profilleri ve hastaların sorduğu sorulara verilen cevapları içeren bir platformdur. Bu web sitesi, doktorların uzmanlık alanları, eğitim geçmişleri, çalıştıkları hastaneler ve hastaların sorduğu sorulara verdikleri cevaplar gibi bilgileri içermektedir. Bu bilgiler, BDM'lerin sağlık alanındaki terminolojiyi, hastalık sınıflandırmalarını ve hasta-doktor iletişim dinamiklerini öğrenmeleri için gerekli verileri sağlayacaktır. Veri seti, web scraping yöntemleri kullanılarak elde edilmiş ve belirli bir formata dönüştürülmüştür. Bu formatta, her bir doktor profili ve hastaların sorduğu sorulara verilen cevaplar ayrı ayrı kaydedilmiştir. Bu veri seti, BDM'lerin eğitiminde kullanılmak üzere hazırlanmıştır. Veri toplama süreci, web scraping yöntemleri kullanılarak gerçekleştirilecektir.