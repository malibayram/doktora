\documentclass[conference]{IEEEtran}
\usepackage{cite}
\usepackage{url}
\usepackage[utf8]{inputenc}
\usepackage[T1]{fontenc}
\usepackage[english, turkish,shorthands=:!]{babel}

\usepackage{array}

\newcolumntype{Y}{>{\centering\arraybackslash}X}
\newcolumntype{C}{>{\centering\arraybackslash}m{\dimexpr.5\linewidth-2\tabcolsep}}
\setlength{\tabcolsep}{10pt} % Adjust horizontal padding to 10pt
\renewcommand{\arraystretch}{1.5} % Increase the row height

\def\BibTeX{{\rm B\kern-.05em{\sc i\kern-.025em b}\kern-.08em
    T\kern-.1667em\lower.7ex\hbox{E}\kern-.125emX}}


% Anahtar kelimeler için özel ortam
\newenvironment{anahtarkelimeler}{
    \small
    \textbf{Anahtar kelimeler:}
}{}

% ingilizce Anahtar kelimeler için özel ortam
\newenvironment{keywords}{
    \small
    \textbf{Keywords:}
}{}

\newenvironment{poliabstract}[1]
  {\renewcommand{\abstractname}{#1}\begin{abstract}}
  {\end{abstract}}


\begin{document}

\title{Sağlık Alanında Büyük Dil Modellerinin Adaptasyonu ve Veri Toplama ve Hazırlama Süreci: Bir Pratik Örnek}


\author{
    \IEEEauthorblockN{
        M.Ali Bayram\IEEEauthorrefmark{1},
        Banu Diri\IEEEauthorrefmark{1},
        Savaş Yıldırım\IEEEauthorrefmark{2},
    }
    \IEEEauthorblockA{\IEEEauthorrefmark{1}Department of Computer Engineering, Yildiz Technical University, Istanbul, Turkey}
    \IEEEauthorblockA{\IEEEauthorrefmark{2}Department of Computer Engineering, Bilgi University, Istanbul, Turkey}
}

\maketitle

\begin{abstract}
    \sloppy
  Büyük dil modelleri (BDM), tıbbi bilgiye erişimi iyileştirmek, hastalarla iletişimi güçlendirmek ve yeni tedaviler geliştirmek gibi potansiyelleri ile sağlık alanında devrim yaratma potansiyeline sahiptir. Ancak, BDM'lerin sağlık alanında etkili ve güvenli bir şekilde kullanılabilmesi için yüksek kaliteli veri toplamak ve hazırlamak gereklidir. Bu makalede, doktorlar tarafından hastaların sorduğu sorulara cevaplar verilen ve herkese açık olarak paylaşılan bir web sitesinden elde edilen doktor anonim profilleri ve soru-cevap verileri kullanılarak BDM adaptasyonu için veri toplama ve hazırlama süreci ele alınmıştır. Veri toplama, veri birleştirme, boş değerlerin işlenmesi, veri tipi dönüşümü, veri temizleme, veri kalite kontrolü, BDM eğitimine hazırlık ve metin ön işleme gibi adımlar detaylı olarak açıklanmıştır. Hazırlanan veriler kullanılarak Meta şirketinin geliştirdiği LLAMA 3 modeli ve YTÜ COSMOS yapay zeka araştırma grubunun geliştirdiği cosmosGPT v0.1 modeli üzerinde fine-tuning işlemi gerçekleştirilmiştir. Böylece modeller, sağlık alanında Türkçe sorulan sorulara daha iyi cevaplar verebilmeye başlamıştır. Bu çalışma, BDM'lerin sağlık alanında kullanımı için kaliteli veri toplama ve hazırlamanın önemini vurgulamaktadır.


\end{abstract}

\begin{anahtarkelimeler}Büyük Dil Modelleri, Sağlık Alanı, Veri Toplama, Veri Hazırlama, Kalite Kontrol, BDM Eğitimi, Fine-tuning.
\end{anahtarkelimeler}

%put a blank line after \end{anahtarkelimeler} for the English abstract
\hspace{0.15in}


\selectlanguage{english}
\begin{abstract}
    \sloppy
Large language models (LLMs) have the potential to revolutionize the field of healthcare with their capabilities to improve access to medical information, strengthen communication with patients, and develop new treatments. However, for LLMs to be effectively and safely used in the healthcare field, it is necessary to collect and prepare high-quality data. This article discusses the process of data collection and preparation for LLM adaptation using doctor anonymous profiles and question-answer data obtained from a website where doctors answer questions asked by patients and shared publicly. Steps such as data collection, data merging, handling missing values, data type conversion, data cleaning, data quality control, preparation for LLM training, and text preprocessing are detailed. The prepared data is used for fine-tuning the LLAMA 3 model developed by Meta company and the cosmosGPT v0.1 model developed by YTU COSMOS artificial intelligence research group. As a result, the models have started to provide better answers to questions asked in Turkish in the healthcare field. This study highlights the importance of collecting and preparing quality data for the use of LLMs in healthcare.
\end{abstract}
\begin{keywords}
    Large Language Models, Healthcare Field, Data Collection, Data Preparation, Quality Control, LLM Training, Fine-tuning.
\end{keywords}  

\section{Giriş}

Sağlık hizmetleri, insan yaşamının en önemli ve hassas alanlarından biridir. Sağlıklı bir yaşam sürdürebilmek için doğru ve zamanında tıbbi bilgiye erişim, hastalıkların doğru teşhisi ve etkili tedavi yöntemlerinin uygulanması büyük önem taşımaktadır. Son yıllarda teknolojinin hızla gelişmesi, sağlık hizmetlerinin sunumunda, teşhis ve tedavi süreçlerinde yeni ve heyecan verici olanaklar yaratmaktadır. Yapay zeka (YZ), bu dönüşümün ön saflarında yer alan teknolojilerden biridir. YZ, karmaşık tıbbi verileri analiz ederek, hastalıkları teşhis etmede, kişiselleştirilmiş tedavi planları oluşturmada ve hatta yeni ilaçlar keşfetmede kullanılabilme potansiyeline sahiptir.
\linebreak
Bu potansiyelin en önemli temsilcilerinden biri de Büyük Dil Modelleri'dir (BDM). BDM'ler, devasa metin veri kümeleri üzerinde eğitilmiş derin öğrenme algoritmalarıdır. Bu modeller, doğal dili anlama, yorumlama ve üretme konusunda son derece yeteneklidirler. İnsanlar gibi metinleri okuyabilir, yazabilir, özetleyebilir ve hatta farklı diller arasında çeviri yapabilirler.
\linebreak
Sağlık alanında BDM'lerin potansiyel uygulamaları oldukça geniştir. Örneğin:
\linebreak
\textbf{Tıbbi Bilgiye Erişim:} Hastalar, BDM'ler aracılığıyla tıbbi bilgilerine kolayca erişebilir, hastalıkları hakkında bilgi edinebilir, semptomlarını değerlendirebilir ve tedavi seçenekleri hakkında bilgi alabilirler.
\linebreak
\textbf{Hasta-Doktor İletişimi:} BDM'ler, hasta-doktor iletişimini kolaylaştırmak için kullanılabilir. Örneğin, hastaların sorularını yanıtlayarak, randevu planlamasına yardımcı olarak ve doktorlara hastaların tıbbi geçmişleri hakkında bilgi sağlayarak iletişim süreçlerini daha verimli hale getirebilirler.
\linebreak
\textbf{Tıbbi Teşhis:} BDM'ler, hastaların tıbbi kayıtlarını, semptomlarını ve tıbbi geçmişlerini analiz ederek doktorlara teşhis koymada yardımcı olabilirler.
\textbf{Tedavi Planlaması:} BDM'ler, hastaların tıbbi geçmişlerini ve semptomlarını analiz ederek, kişiselleştirilmiş tedavi planları oluşturabilir ve tedavi süreçlerini optimize edebilirler.
\linebreak
Ancak, BDM'lerin sağlık alanındaki tüm bu potansiyel faydalarını gerçekleştirebilmesi için, bu alana özgü yüksek kaliteli verilerle eğitilmeleri gerekmektedir. Tıbbi metinler, karmaşık terminolojiye, hastalık sınıflandırmalarına, tedavi yöntemlerine ve hasta-doktor iletişim dinamiklerine sahiptir. BDM'lerin bu alandaki verileri doğru bir şekilde anlayabilmesi ve işleyebilmesi için, bu verilere özgü bir şekilde eğitilmeleri gerekmektedir.
\linebreak
Bu noktada, veri toplama ve hazırlama süreçleri büyük önem kazanmaktadır. BDM'lerin sağlık alanına adaptasyonu, bu modellerin sadece genel dil yapısını değil, aynı zamanda sağlık alanına özgü terminolojiyi, hastalık sınıflandırmalarını, tedavi yöntemlerini ve hasta-doktor iletişim dinamiklerini anlamalarını gerektirir. Bu nedenle, BDM'lerin sağlık alanında etkili bir şekilde kullanılabilmesi için, bu alana özgü verilerin toplanması, temizlenmesi, yapılandırılması ve özenle hazırlanması gerekmektedir.
\linebreak
Bu çalışmanın temel amacı, sağlık alanında özelleştirilmiş bir BDM oluşturmak için gerekli veri toplama ve hazırlama süreçlerini detaylı bir şekilde açıklamak ve bu sürecin, BDM'lerin sağlık alanındaki performansını nasıl etkilediğini göstermektir. Bu amaçla iki farklı BDM modeli kullanılacaktır: Meta tarafından geliştirilen LLAMA 3 ve YTÜ COSMOS yapay zeka araştırma grubunun geliştirdiği cosmosGPT v0.1 \cite{kesgin2024introducing}. Bu modellerin farklı yetenekleri ve eğitim verileri, BDM adaptasyon süreci ve sağlık alanına özgü verilerin etkisini daha iyi anlamamızı sağlayacaktır.
\linebreak
\textbf{LLAMA 3}, Meta tarafından geliştirilen açık kaynak kodlu bir BDM'dir. LLAMA 3, geniş bir metin veri kümesi üzerinde eğitilmiş olup, doğal dil işleme görevlerinde etkileyici bir performans sergilemektedir. Ancak, genel amaçlı bir model olması nedeniyle, sağlık alanına özgü terminoloji, hastalık sınıflandırmaları, tedavi yöntemleri ve hasta-doktor iletişim dinamikleri konusunda yeterli bilgiye sahip değildir.
\linebreak
\textbf{cosmosGPT v0.1} ise, YTÜ COSMOS yapay zeka araştırma grubu tarafından geliştirilmiş ve özellikle Türkçe metinler üzerinde eğitilmiş bir BDM'dir. Bu model, Türkçe dil yapısını ve yaygın kullanılan Türkçe kelime dağarcığını iyi bir şekilde anlamakta ve işlemektedir. Ancak, LLAMA 3 gibi, cosmosGPT v0.1 de sağlık alanına özgü bilgiler konusunda eksikliklere sahiptir.
\linebreak
Bu çalışmada, LLAMA 3 ve cosmosGPT v0.1 modellerinin sağlık alanına adaptasyonu için fine-tuning yöntemi kullanılacaktır. Fine-tuning, önceden eğitilmiş bir BDM'nin, yeni bir göreve veya alana özgü verilerle ek olarak eğitilmesi işlemidir. Bu sayede, model belirli bir alandaki performansını artırabilir. Bu çalışmada kullanılan sağlık alanına özgü veri seti, doktorlar tarafından hastaların sorduğu sorulara verilen cevaplardan oluşmaktadır. Bu veri seti, BDM'lerin sağlık alanındaki terminolojiyi, hastalık sınıflandırmalarını ve hasta-doktor iletişim dinamiklerini öğrenmelerini sağlayacaktır.
\linebreak
Fine-tuning işlemi sırasında, LLAMA 3 ve cosmosGPT v0.1 modelleri bu veri seti üzerinde eğitilecek ve sağlık alanına özgü sorulara daha doğru ve alakalı cevaplar vermesi hedeflenecektir. Modellerin performansı, eğitim ve test veri setleri kullanılarak değerlendirilecektir. Fine-tuning işlemi sonucunda, LLAMA 3 ve cosmosGPT v0.1 modellerinin sağlık alanındaki performansının artması ve sağlık hizmetleri alanında kullanılabilecek özelleştirilmiş BDM'ler olabilme potansiyelleri gözlemlenecektir.
\linebreak
Başarımın test edilmesi için ayrıca küçük bir veri seti üzerinden modellerin verdiği cevaplar sağlık profesyonelleri tarafından değerlendirilecektir. Bu değerlendirme sonucunda, modellerin sağlık alanındaki terminolojiyi, hastalık sınıflandırmalarını ve hasta-doktor iletişim dinamiklerini ne kadar iyi anladığı ve doğru cevaplar verdiği belirlenecektir.
\linebreak
Bu çalışma, BDM'lerin sağlık hizmetlerinde kullanımı için atılan önemli bir adımdır. BDM'lerin sağlık alanındaki potansiyel faydaları çok çeşitlidir. Ancak, bu potansiyeli tam olarak gerçekleştirebilmek için, doğru ve etkili veri toplama ve hazırlama süreçlerine dikkat edilmesi gerekmektedir. Bu çalışma, bu süreçleri detaylı bir şekilde açıklayarak, BDM'lerin sağlık alanındaki uygulamalarına yönelik araştırmalara katkı sağlamayı amaçlamaktadır.

\section{Veri Toplama ve Hazırlama Süreci}
Veri toplama süreci, doktorların hastalara verdiği cevapları içeren doktor profilleri ve soru-cevap verilerini içeren bir veri seti üzerinde gerçekleştirilecektir. Bu veri seti, doktorsitesi.com adlı bir web sitesinden elde edilmiştir. Bu web sitesi, doktorların doğrulanmış profilleri ve hastaların sorduğu sorulara verilen cevapları içeren bir platformdur. Bu web sitesi, doktorların uzmanlık alanları, eğitim geçmişleri, çalıştıkları hastaneler ve hastaların sorduğu sorulara verdikleri cevaplar gibi bilgileri içermektedir. Bu bilgiler, BDM'lerin sağlık alanındaki terminolojiyi, hastalık sınıflandırmalarını ve hasta-doktor iletişim dinamiklerini öğrenmeleri için gerekli verileri sağlayacaktır. Veri seti, web scraping yöntemleri kullanılarak elde edilmiş ve belirli bir formata dönüştürülmüştür. Bu formatta, her bir doktor profili ve hastaların sorduğu sorulara verilen cevaplar ayrı ayrı kaydedilmiştir. Bu veri seti, BDM'lerin eğitiminde kullanılmak üzere hazırlanmıştır. Veri toplama süreci, web scraping yöntemleri kullanılarak gerçekleştirilecektir.

\section{Yöntemler}
\sloppy
Lorem ipsum dolor sit amet, consectetur adipiscing elit

\section{Bulgular}
\sloppy
Lorem ipsum dolor sit amet, consectetur adipiscing elit

\section{Tartışma}
\sloppy
Lorem ipsum dolor sit amet, consectetur adipiscing elit

\section{Sonuç}
\sloppy
Lorem ipsum dolor sit amet, consectetur adipiscing elit

\bibliographystyle{ieeetr}
\bibliography{references}

\end{document}
