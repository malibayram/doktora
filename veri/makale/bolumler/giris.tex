
Sağlık hizmetleri, insan yaşamının en önemli ve hassas alanlarından biridir. Sağlıklı bir yaşam sürdürebilmek için doğru ve zamanında tıbbi bilgiye erişim, hastalıkların doğru teşhisi ve etkili tedavi yöntemlerinin uygulanması büyük önem taşımaktadır. Son yıllarda teknolojinin hızla gelişmesi, sağlık hizmetlerinin sunumunda, teşhis ve tedavi süreçlerinde yeni ve heyecan verici olanaklar yaratmaktadır. Yapay zeka (YZ), bu dönüşümün ön saflarında yer alan teknolojilerden biridir. YZ, karmaşık tıbbi verileri analiz ederek, hastalıkları teşhis etmede, kişiselleştirilmiş tedavi planları oluşturmada ve hatta yeni ilaçlar keşfetmede kullanılabilme potansiyeline sahiptir.
\linebreak
Bu potansiyelin en önemli temsilcilerinden biri de Büyük Dil Modelleri'dir (BDM). BDM'ler, devasa metin veri kümeleri üzerinde eğitilmiş derin öğrenme algoritmalarıdır. Bu modeller, doğal dili anlama, yorumlama ve üretme konusunda son derece yeteneklidirler. İnsanlar gibi metinleri okuyabilir, yazabilir, özetleyebilir ve hatta farklı diller arasında çeviri yapabilirler.
\linebreak
Sağlık alanında BDM'lerin potansiyel uygulamaları oldukça geniştir. Örneğin:
\linebreak
\textbf{Tıbbi Bilgiye Erişim:} Hastalar, BDM'ler aracılığıyla tıbbi bilgilerine kolayca erişebilir, hastalıkları hakkında bilgi edinebilir, semptomlarını değerlendirebilir ve tedavi seçenekleri hakkında bilgi alabilirler.
\linebreak
\textbf{Hasta-Doktor İletişimi:} BDM'ler, hasta-doktor iletişimini kolaylaştırmak için kullanılabilir. Örneğin, hastaların sorularını yanıtlayarak, randevu planlamasına yardımcı olarak ve doktorlara hastaların tıbbi geçmişleri hakkında bilgi sağlayarak iletişim süreçlerini daha verimli hale getirebilirler.
\linebreak
\textbf{Tıbbi Teşhis:} BDM'ler, hastaların tıbbi kayıtlarını, semptomlarını ve tıbbi geçmişlerini analiz ederek doktorlara teşhis koymada yardımcı olabilirler.
\textbf{Tedavi Planlaması:} BDM'ler, hastaların tıbbi geçmişlerini ve semptomlarını analiz ederek, kişiselleştirilmiş tedavi planları oluşturabilir ve tedavi süreçlerini optimize edebilirler.
\linebreak
Ancak, BDM'lerin sağlık alanındaki tüm bu potansiyel faydalarını gerçekleştirebilmesi için, bu alana özgü yüksek kaliteli verilerle eğitilmeleri gerekmektedir. Tıbbi metinler, karmaşık terminolojiye, hastalık sınıflandırmalarına, tedavi yöntemlerine ve hasta-doktor iletişim dinamiklerine sahiptir. BDM'lerin bu alandaki verileri doğru bir şekilde anlayabilmesi ve işleyebilmesi için, bu verilere özgü bir şekilde eğitilmeleri gerekmektedir.
\linebreak
Bu noktada, veri toplama ve hazırlama süreçleri büyük önem kazanmaktadır. BDM'lerin sağlık alanına adaptasyonu, bu modellerin sadece genel dil yapısını değil, aynı zamanda sağlık alanına özgü terminolojiyi, hastalık sınıflandırmalarını, tedavi yöntemlerini ve hasta-doktor iletişim dinamiklerini anlamalarını gerektirir. Bu nedenle, BDM'lerin sağlık alanında etkili bir şekilde kullanılabilmesi için, bu alana özgü verilerin toplanması, temizlenmesi, yapılandırılması ve özenle hazırlanması gerekmektedir.
\linebreak
Bu çalışmanın temel amacı, sağlık alanında özelleştirilmiş bir BDM oluşturmak için gerekli veri toplama ve hazırlama süreçlerini detaylı bir şekilde açıklamak ve bu sürecin, BDM'lerin sağlık alanındaki performansını nasıl etkilediğini göstermektir. Bu amaçla iki farklı BDM modeli kullanılacaktır: Meta tarafından geliştirilen LLAMA 3 ve YTÜ COSMOS yapay zeka araştırma grubunun geliştirdiği cosmosGPT v0.1 \cite{kesgin2024introducing}. Bu modellerin farklı yetenekleri ve eğitim verileri, BDM adaptasyon süreci ve sağlık alanına özgü verilerin etkisini daha iyi anlamamızı sağlayacaktır.
\linebreak
\textbf{LLAMA 3}, Meta tarafından geliştirilen açık kaynak kodlu bir BDM'dir. LLAMA 3, geniş bir metin veri kümesi üzerinde eğitilmiş olup, doğal dil işleme görevlerinde etkileyici bir performans sergilemektedir. Ancak, genel amaçlı bir model olması nedeniyle, sağlık alanına özgü terminoloji, hastalık sınıflandırmaları, tedavi yöntemleri ve hasta-doktor iletişim dinamikleri konusunda yeterli bilgiye sahip değildir.
\linebreak
\textbf{cosmosGPT v0.1} ise, YTÜ COSMOS yapay zeka araştırma grubu tarafından geliştirilmiş ve özellikle Türkçe metinler üzerinde eğitilmiş bir BDM'dir. Bu model, Türkçe dil yapısını ve yaygın kullanılan Türkçe kelime dağarcığını iyi bir şekilde anlamakta ve işlemektedir. Ancak, LLAMA 3 gibi, cosmosGPT v0.1 de sağlık alanına özgü bilgiler konusunda eksikliklere sahiptir.
\linebreak
Bu çalışmada, LLAMA 3 ve cosmosGPT v0.1 modellerinin sağlık alanına adaptasyonu için fine-tuning yöntemi kullanılacaktır. Fine-tuning, önceden eğitilmiş bir BDM'nin, yeni bir göreve veya alana özgü verilerle ek olarak eğitilmesi işlemidir. Bu sayede, model belirli bir alandaki performansını artırabilir. Bu çalışmada kullanılan sağlık alanına özgü veri seti, doktorlar tarafından hastaların sorduğu sorulara verilen cevaplardan oluşmaktadır. Bu veri seti, BDM'lerin sağlık alanındaki terminolojiyi, hastalık sınıflandırmalarını ve hasta-doktor iletişim dinamiklerini öğrenmelerini sağlayacaktır.
\linebreak
Fine-tuning işlemi sırasında, LLAMA 3 ve cosmosGPT v0.1 modelleri bu veri seti üzerinde eğitilecek ve sağlık alanına özgü sorulara daha doğru ve alakalı cevaplar vermesi hedeflenecektir. Modellerin performansı, eğitim ve test veri setleri kullanılarak değerlendirilecektir. Fine-tuning işlemi sonucunda, LLAMA 3 ve cosmosGPT v0.1 modellerinin sağlık alanındaki performansının artması ve sağlık hizmetleri alanında kullanılabilecek özelleştirilmiş BDM'ler olabilme potansiyelleri gözlemlenecektir.
\linebreak
Başarımın test edilmesi için ayrıca küçük bir veri seti üzerinden modellerin verdiği cevaplar sağlık profesyonelleri tarafından değerlendirilecektir. Bu değerlendirme sonucunda, modellerin sağlık alanındaki terminolojiyi, hastalık sınıflandırmalarını ve hasta-doktor iletişim dinamiklerini ne kadar iyi anladığı ve doğru cevaplar verdiği belirlenecektir.
\linebreak
Bu çalışma, BDM'lerin sağlık hizmetlerinde kullanımı için atılan önemli bir adımdır. BDM'lerin sağlık alanındaki potansiyel faydaları çok çeşitlidir. Ancak, bu potansiyeli tam olarak gerçekleştirebilmek için, doğru ve etkili veri toplama ve hazırlama süreçlerine dikkat edilmesi gerekmektedir. Bu çalışma, bu süreçleri detaylı bir şekilde açıklayarak, BDM'lerin sağlık alanındaki uygulamalarına yönelik araştırmalara katkı sağlamayı amaçlamaktadır.
