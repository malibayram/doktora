\sloppy
  Büyük dil modelleri (BDM), tıbbi bilgiye erişimi iyileştirmek, hastalarla iletişimi güçlendirmek ve yeni tedaviler geliştirmek gibi potansiyelleri ile sağlık alanında devrim yaratma potansiyeline sahiptir. Ancak, BDM'lerin sağlık alanında etkili ve güvenli bir şekilde kullanılabilmesi için yüksek kaliteli veri toplamak ve hazırlamak gereklidir. Bu makalede, doktorlar tarafından hastaların sorduğu sorulara cevaplar verilen ve herkese açık olarak paylaşılan bir web sitesinden elde edilen doktor anonim profilleri ve soru-cevap verileri kullanılarak BDM adaptasyonu için veri toplama ve hazırlama süreci ele alınmıştır. Veri toplama, veri birleştirme, boş değerlerin işlenmesi, veri tipi dönüşümü, veri temizleme, veri kalite kontrolü, BDM eğitimine hazırlık ve metin ön işleme gibi adımlar detaylı olarak açıklanmıştır. Hazırlanan veriler kullanılarak Meta şirketinin geliştirdiği LLAMA 3 modeli ve YTÜ COSMOS yapay zeka araştırma grubunun geliştirdiği cosmosGPT v0.1 modeli üzerinde fine-tuning işlemi gerçekleştirilmiştir. Böylece modeller, sağlık alanında Türkçe sorulan sorulara daha iyi cevaplar verebilmeye başlamıştır. Bu çalışma, BDM'lerin sağlık alanında kullanımı için kaliteli veri toplama ve hazırlamanın önemini vurgulamaktadır.

