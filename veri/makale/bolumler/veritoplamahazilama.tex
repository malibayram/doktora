Veri toplama süreci, doktorların hastalara verdiği cevapları içeren doktor profilleri ve soru-cevap verilerini içeren bir veri seti üzerinde gerçekleştirilecektir. Bu veri seti, doktorsitesi.com adlı bir web sitesinden elde edilmiştir. Bu web sitesi, doktorların doğrulanmış profilleri ve hastaların sorduğu sorulara verilen cevapları içeren bir platformdur. Bu web sitesi, doktorların uzmanlık alanları, eğitim geçmişleri, çalıştıkları hastaneler ve hastaların sorduğu sorulara verdikleri cevaplar gibi bilgileri içermektedir. Bu bilgiler, BDM'lerin sağlık alanındaki terminolojiyi, hastalık sınıflandırmalarını ve hasta-doktor iletişim dinamiklerini öğrenmeleri için gerekli verileri sağlayacaktır. Veri seti, web scraping yöntemleri kullanılarak elde edilmiş ve belirli bir formata dönüştürülmüştür. Bu formatta, her bir doktor profili ve hastaların sorduğu sorulara verilen cevaplar ayrı ayrı kaydedilmiştir. Bu veri seti, BDM'lerin eğitiminde kullanılmak üzere hazırlanmıştır. Veri toplama süreci, web scraping yöntemleri kullanılarak gerçekleştirilecektir.
\subsection{Verilerin Toplanması için Gerekli Araçların Hazırlanması}
\textbf{Kullanılan Bilgisayar:} Macbook Pro 16-inch, 2023, Apple M2 Max chip, 64 GB RAM, 1 TB SSD depolama, macOS Version 14.5 (23F79) işletim sistemi.
\newline
\textbf{Veri Kaynağı:} doktorsitesi.com
\newline
\textbf{Programlama Dili:} Python 3.9.6
\newline
\textbf{Kodlama Ortamı:} Visual Studio Code Versiyon: 1.89.1 (MacOS) ve Eklenti olarak Jupiter Notebook Versiyon: 2024.4.0
\newline
\textbf{Kullanılan Kütüphaneler:}
\begin{itemize}
  \hbadness=99999
  \item requests: Internete açılan bağlantılar üzerinden veri alışverişi yapmak için kullanılacak.
  \item BeautifulSoup: Web sayfalarından alınan HTML verilerini parse etmek ve analiz etmek için kullanılacak.
  \item asyncio: Asenkron programlama yapmak için kullanılacak.
  \item aiohttp: Asenkron HTTP istekleri yapmak için kullanılacak.
  \item pandas: Veri analizi ve işleme için kullanılacak.
  \item json: JSON verileri işlemek için kullanılacak.
\end{itemize}
